\documentclass[]{book}
\usepackage{lmodern}
\usepackage{amssymb,amsmath}
\usepackage{ifxetex,ifluatex}
\usepackage{fixltx2e} % provides \textsubscript
\ifnum 0\ifxetex 1\fi\ifluatex 1\fi=0 % if pdftex
  \usepackage[T1]{fontenc}
  \usepackage[utf8]{inputenc}
  \usepackage{eurosym}
\else % if luatex or xelatex
  \ifxetex
    \usepackage{mathspec}
  \else
    \usepackage{fontspec}
  \fi
  \defaultfontfeatures{Ligatures=TeX,Scale=MatchLowercase}
  \newcommand{\euro}{€}
\fi
% use upquote if available, for straight quotes in verbatim environments
\IfFileExists{upquote.sty}{\usepackage{upquote}}{}
% use microtype if available
\IfFileExists{microtype.sty}{%
\usepackage{microtype}
\UseMicrotypeSet[protrusion]{basicmath} % disable protrusion for tt fonts
}{}
\usepackage[margin=1in]{geometry}
\usepackage{hyperref}
\hypersetup{unicode=true,
            pdftitle={My dissertation},
            pdfauthor={Tristan Mahr},
            pdfborder={0 0 0},
            breaklinks=true}
\urlstyle{same}  % don't use monospace font for urls
\usepackage{natbib}
\bibliographystyle{apalike}
\usepackage{color}
\usepackage{fancyvrb}
\newcommand{\VerbBar}{|}
\newcommand{\VERB}{\Verb[commandchars=\\\{\}]}
\DefineVerbatimEnvironment{Highlighting}{Verbatim}{commandchars=\\\{\}}
% Add ',fontsize=\small' for more characters per line
\usepackage{framed}
\definecolor{shadecolor}{RGB}{248,248,248}
\newenvironment{Shaded}{\begin{snugshade}}{\end{snugshade}}
\newcommand{\KeywordTok}[1]{\textcolor[rgb]{0.13,0.29,0.53}{\textbf{#1}}}
\newcommand{\DataTypeTok}[1]{\textcolor[rgb]{0.13,0.29,0.53}{#1}}
\newcommand{\DecValTok}[1]{\textcolor[rgb]{0.00,0.00,0.81}{#1}}
\newcommand{\BaseNTok}[1]{\textcolor[rgb]{0.00,0.00,0.81}{#1}}
\newcommand{\FloatTok}[1]{\textcolor[rgb]{0.00,0.00,0.81}{#1}}
\newcommand{\ConstantTok}[1]{\textcolor[rgb]{0.00,0.00,0.00}{#1}}
\newcommand{\CharTok}[1]{\textcolor[rgb]{0.31,0.60,0.02}{#1}}
\newcommand{\SpecialCharTok}[1]{\textcolor[rgb]{0.00,0.00,0.00}{#1}}
\newcommand{\StringTok}[1]{\textcolor[rgb]{0.31,0.60,0.02}{#1}}
\newcommand{\VerbatimStringTok}[1]{\textcolor[rgb]{0.31,0.60,0.02}{#1}}
\newcommand{\SpecialStringTok}[1]{\textcolor[rgb]{0.31,0.60,0.02}{#1}}
\newcommand{\ImportTok}[1]{#1}
\newcommand{\CommentTok}[1]{\textcolor[rgb]{0.56,0.35,0.01}{\textit{#1}}}
\newcommand{\DocumentationTok}[1]{\textcolor[rgb]{0.56,0.35,0.01}{\textbf{\textit{#1}}}}
\newcommand{\AnnotationTok}[1]{\textcolor[rgb]{0.56,0.35,0.01}{\textbf{\textit{#1}}}}
\newcommand{\CommentVarTok}[1]{\textcolor[rgb]{0.56,0.35,0.01}{\textbf{\textit{#1}}}}
\newcommand{\OtherTok}[1]{\textcolor[rgb]{0.56,0.35,0.01}{#1}}
\newcommand{\FunctionTok}[1]{\textcolor[rgb]{0.00,0.00,0.00}{#1}}
\newcommand{\VariableTok}[1]{\textcolor[rgb]{0.00,0.00,0.00}{#1}}
\newcommand{\ControlFlowTok}[1]{\textcolor[rgb]{0.13,0.29,0.53}{\textbf{#1}}}
\newcommand{\OperatorTok}[1]{\textcolor[rgb]{0.81,0.36,0.00}{\textbf{#1}}}
\newcommand{\BuiltInTok}[1]{#1}
\newcommand{\ExtensionTok}[1]{#1}
\newcommand{\PreprocessorTok}[1]{\textcolor[rgb]{0.56,0.35,0.01}{\textit{#1}}}
\newcommand{\AttributeTok}[1]{\textcolor[rgb]{0.77,0.63,0.00}{#1}}
\newcommand{\RegionMarkerTok}[1]{#1}
\newcommand{\InformationTok}[1]{\textcolor[rgb]{0.56,0.35,0.01}{\textbf{\textit{#1}}}}
\newcommand{\WarningTok}[1]{\textcolor[rgb]{0.56,0.35,0.01}{\textbf{\textit{#1}}}}
\newcommand{\AlertTok}[1]{\textcolor[rgb]{0.94,0.16,0.16}{#1}}
\newcommand{\ErrorTok}[1]{\textcolor[rgb]{0.64,0.00,0.00}{\textbf{#1}}}
\newcommand{\NormalTok}[1]{#1}
\usepackage{longtable,booktabs}
\usepackage{graphicx,grffile}
\makeatletter
\def\maxwidth{\ifdim\Gin@nat@width>\linewidth\linewidth\else\Gin@nat@width\fi}
\def\maxheight{\ifdim\Gin@nat@height>\textheight\textheight\else\Gin@nat@height\fi}
\makeatother
% Scale images if necessary, so that they will not overflow the page
% margins by default, and it is still possible to overwrite the defaults
% using explicit options in \includegraphics[width, height, ...]{}
\setkeys{Gin}{width=\maxwidth,height=\maxheight,keepaspectratio}
\IfFileExists{parskip.sty}{%
\usepackage{parskip}
}{% else
\setlength{\parindent}{0pt}
\setlength{\parskip}{6pt plus 2pt minus 1pt}
}
\setlength{\emergencystretch}{3em}  % prevent overfull lines
\providecommand{\tightlist}{%
  \setlength{\itemsep}{0pt}\setlength{\parskip}{0pt}}
\setcounter{secnumdepth}{5}
% Redefines (sub)paragraphs to behave more like sections
\ifx\paragraph\undefined\else
\let\oldparagraph\paragraph
\renewcommand{\paragraph}[1]{\oldparagraph{#1}\mbox{}}
\fi
\ifx\subparagraph\undefined\else
\let\oldsubparagraph\subparagraph
\renewcommand{\subparagraph}[1]{\oldsubparagraph{#1}\mbox{}}
\fi

%%% Use protect on footnotes to avoid problems with footnotes in titles
\let\rmarkdownfootnote\footnote%
\def\footnote{\protect\rmarkdownfootnote}

%%% Change title format to be more compact
\usepackage{titling}

% Create subtitle command for use in maketitle
\newcommand{\subtitle}[1]{
  \posttitle{
    \begin{center}\large#1\end{center}
    }
}

\setlength{\droptitle}{-2em}
  \title{My dissertation}
  \pretitle{\vspace{\droptitle}\centering\huge}
  \posttitle{\par}
  \author{Tristan Mahr}
  \preauthor{\centering\large\emph}
  \postauthor{\par}
  \predate{\centering\large\emph}
  \postdate{\par}
  \date{2017-08-15}

\usepackage{booktabs}
\usepackage{amsthm}
\makeatletter
\def\thm@space@setup{%
  \thm@preskip=8pt plus 2pt minus 4pt
  \thm@postskip=\thm@preskip
}
\makeatother

\usepackage{amsthm}
\newtheorem{theorem}{Theorem}[chapter]
\newtheorem{lemma}{Lemma}[chapter]
\theoremstyle{definition}
\newtheorem{definition}{Definition}[chapter]
\newtheorem{corollary}{Corollary}[chapter]
\newtheorem{proposition}{Proposition}[chapter]
\theoremstyle{definition}
\newtheorem{example}{Example}[chapter]
\theoremstyle{remark}
\newtheorem*{remark}{Remark}
\begin{document}
\maketitle

{
\setcounter{tocdepth}{1}
\tableofcontents
}
\chapter{{[}demo{]} Prerequisites}\label{demo-prerequisites}

This is a \emph{sample} book written in \textbf{Markdown}. You can use
anything that Pandoc's Markdown supports, e.g., a math equation
\(a^2 + b^2 = c^2\).

For now, you have to install the development versions of
\textbf{bookdown} from Github:

\begin{Shaded}
\begin{Highlighting}[]
\NormalTok{devtools}\OperatorTok{::}\KeywordTok{install_github}\NormalTok{(}\StringTok{"rstudio/bookdown"}\NormalTok{)}
\end{Highlighting}
\end{Shaded}

Remember each Rmd file contains one and only one chapter, and a chapter
is defined by the first-level heading \texttt{\#}.

To compile this example to PDF, you need to install XeLaTeX.

\part*{Prospectus}\label{part-prospectus}
\addcontentsline{toc}{part}{Prospectus}

\chapter{Front Matter}\label{front-matter}

\subsection*{About This Document}\label{about-this-document}
\addcontentsline{toc}{subsection}{About This Document}

This document outlines the research questions, data, and methods for my
dissertation. This proposal started out as a grant-writing project, so
it has some of the touchstones of NIH F31 grant (Specific Aims,
Significance, Approach), but these sections have been expanded
considerably.

\subsection*{Dissertation Committee
Members}\label{dissertation-committee-members}
\addcontentsline{toc}{subsection}{Dissertation Committee Members}

\begin{itemize}
\tightlist
\item
  Jan Edwards, primary mentor and chair, Department of Hearing and
  Speech Sciences, University of Maryland
\item
  Susan Ellis Weismer, official advisor at UW-Madison, Department of
  Communication Sciences and Disorders
\item
  Margarita Kaushanskaya, Department of Communication Sciences and
  Disorders
\item
  Audra Sterling, Department of Communication Sciences and Disorders
\item
  David Kaplan, Department of Educational Psychology
\item
  Bob McMurray, Department of Psychological and Brain Sciences,
  University of Iowa
\end{itemize}

\subsection*{Planned Dissertation
Format}\label{planned-dissertation-format}
\addcontentsline{toc}{subsection}{Planned Dissertation Format}

Three thematically related manuscripts, one for each specific aim, to be
completed by Summer 2018.

\chapter{Specific Aims}\label{specific-aims}

Individual differences in language ability are apparent as soon as
children start talking, but it is difficult to identify children at risk
for language delay or disorder. Recent work suggests word recognition
efficiency---that is, how well children map incoming speech to
words---may help identify early differences in children's language
trajectories. Children learn spoken language by listening to caregivers,
so children who are faster at recognizing words have an advantage for
word learning. This view is borne out by some studies suggesting that
children who are faster at processing words show greater vocabulary
gains months later \citep[e.g.,][]{Weisleder2013}.

We do not know, however, how word recognition itself develops over time
within a child. This is an important open question because word
recognition may provide a key mechanism for understanding how individual
differences emerge in word learning and persist into early language
development. Without a developmental account of word recognition, we
lack the context for understanding individual differences in lexical
processing. Thus, even the big-picture questions are unclear: Do early
differences persist over time so that faster processors remain
relatively fast later in childhood? Or, is such a question ill-posed
because the magnitude of the differences among children shrink with age?
I plan to address this gap in knowledge by analyzing three years of word
recognition data collected in recently completed longitudinal study of
180 children.

In particular, I will examine the development of \emph{familiar word
recognition}, \emph{lexical competition,} and \emph{fast referent
selection} (the ability to map novel words to novel objects in the
moment). Through these analyses, I will develop a fine-grained
description of how the dynamics of word recognition change year over
year, and I will study how differences in word recognition performance
relate to child-level measures (such as vocabulary and speech
perception). I will complement these empirical analyses with
computational cognitive models. With these models, I will simulate the
word recognition data from each year and study how the models need to
change to adapt to children's developing word recognition abilities.
These simulations can identify plausible psychological mechanisms that
underlie changes in word recognition behavior.

\section{Specific Aim 1 (Familiar Word Recognition and Lexical
Competition)}\label{specific-aim-1-familiar-word-recognition-and-lexical-competition}

\textbf{To characterize the development of familiar word recognition and
lexical competition, I will analyze data from a visual world paradigm
experiment, conducted at age 3, age 4, and age 5.}

In these eyetracking experiments, children were presented with four
images of familiar objects and heard a prompt to view one of the images.
The four images included a target word (e.g., \emph{bell}), a
semantically related word (\emph{drum}), a phonologically similar word
(\emph{bee}), and an unrelated word (\emph{swing}). I will use a series
of growth curve analyses to describe how children's familiar word
recognition develop year over year. Of interest is how individual
differences at Year 1 persist into Year 3. I will also analyze how
expressive vocabulary and lexical processing develop together over time.
Lastly, I will examine the children's looks to the distractors to study
the developmental course of lexical competition from similar sounding
and similar meaning words. Changes in sensitivity to competing words can
reveal how lexical competition emerges as a byproduct of learning new
words.

\section{Specific Aim 2 (Referent Selection and
Mispronunciations)}\label{specific-aim-2-referent-selection-and-mispronunciations}

\textbf{To characterize how fast referent selection develops
longitudinally, I will analyze data from a looking-while-listening
mispronunciation experiment, conducted at age 3, age 4, and age 5.}

In these eyetracking experiments \citep[based
on][]{WhiteMorgan2008, MPPaper}, children saw an image of a familiar
object and an unfamiliar object, and they heard either a correct
production of the familiar object (e.g., \emph{soup}), a one-feature
mispronunciation of the familiar object (\emph{shoop}), or a novel word
unrelated to either image (\emph{cheem}). The correct productions test
familiar word recognition and the nonwords test fast referent selection.
The mispronunciations test a child's phonological categories (whether
the child permits, rejects, or equivocates about mispronunciations).

I will use growth curve analyses to study how children's responses to
the three word types change over time. I will examine familiar word
recognition and fast referent selection to determine which feature of
lexical processing better predicts vocabulary growth. I plan to examine
dissociations or asymmetries in these forms of processing within
children as a way to empirically assess the claim that ``novel word
processing (referent selection) is not distinct from familiar word
recognition'' (McMurray, Horst, \& Samuelson, 2012). Finally, I will
examine how individual differences in vocabulary and speech perception
predict responses to mispronunciations and novel words.

\section{Specific Aim 3 (Computational
Modeling)}\label{specific-aim-3-computational-modeling}

\textbf{To identify plausible psychological mechanisms underlying the
development of word recognition, I will simulate the word recognition
data using cognitive computation models.}

The TRACE model of word recognition \citep{TRACE} has been used to
simulate word recognition data from adults \citep{Allopenna1998}, adults
with aphasia \citep{Mirman2011}, toddlers \citep{TRACE_Mispro}, and
adolescents with language impairments \citep{McMurray2010}. In this
model, incoming acoustic information activates perceptual units which in
turn activate phoneme units which in turn activate word units.
Connections are interactive, so the model can accommodate top-down
processing effects and competition among units through inhibition. The
model is controlled by psychological parameters like inhibition
strength, activation decay rates, and lexicon size and composition. The
advantage of using this model is how one can map different behavioral
patterns onto changes in model parameterizations to develop a plausible
psychological account of developmental changes. I will simulate the Year
1 word recognition data in TRACE, and I will study how the model's
parameters need to change in order to accommodate data from Year 2 and
Year 3. I will examine how different model parameters map onto
individual differences in word recognition. For instance, under what
modeling conditions are mispronunciations of familiar words accepted and
do these conditions correspond to child-level differences in vocabulary
or speech perception? These simulations will provide a psychological
account for the developmental trends and individual variation in word
recognition.

\section{Summary}\label{summary}

This project investigates how word recognition develops during the
preschool years. There has been no research studying word recognition
longitudinally after age two. Findings will show how individual
differences in lexical processing change over time and can reveal how
low-level mechanisms underlying word recognition mature longitudinally
in children. These findings will have translational value by studying
processing abilities that subserve word learning and by assessing the
predictive relationships between early word recognition ability and
later language outcomes.

\chapter{Significance}\label{significance}

\section{Public Health Significance~}\label{public-health-significance}

Vocabulary size in preschool is a robust predictor of later language
development, and early language skills predict early literacy skills at
school entry \citep{Morgan2015}. By studying the mechanisms that shape
word learning, we can understand how individual differences in language
ability arise and identify strategies for closing language gaps between
children. Word recognition---the process of mapping incoming speech
sounds to known or novel words---has been shown to predict later
language outcomes. We do not know how this ability develops over time,
and we do not know when word recognition is most predictive of future
outcomes. This project will provide an integrated account of how word
recognition and its relationship with vocabulary size change from age 3
to age 5.

\section{Scientific Significance}\label{scientific-significance}

\subsection{Lexical Processing
Dynamics}\label{lexical-processing-dynamics}

Mature listeners recognize words by continuously evaluating incoming
speech input for possible word matches through lexical competition. The
first part of a word activates multiple candidate words in parallel, and
these candidates compete so that the best-fitting word is recognized.
For example, the onset ``bee'' might activate the candidates \emph{bee},
\emph{beam}, \emph{beetle}, \emph{beak}, \emph{beaker},
\emph{beginning}, and so on, but an additional ``m'' would narrow the
candidates to just \emph{beam}. Semantic relationships also influence
lexical processing, and cascading phonological-semantic effects---e.g.,
where \emph{castle} activates the phonologically similar \emph{candy}
which in turn activates the semantically related
\emph{sweet}â\euro{}''have been demonstrated \citep{Marslen-Wilson1989}.
Both low-level phonetic cues and high-level grammatical, semantic and
pragmatic information can influence this process, but the
\emph{continuous processing of multiple competing candidates} is the
essential dynamic underlying word recognition \citep{Magnuson2013}.

What about young children who know considerably fewer words? Eyetracking
studies with toddlers have suggested a developmental continuity between
toddlers and adult listeners. Children recognize words incrementally
(Swingley, Pinto, \& Fernald, 1999), match truncated words to their
intended referents (Fernald, Swingley, \& Pinto, 2001), and use
information from neighboring words in a sentence to facilitate word
recognition. This information can be grammatical: Lew-Williams and
Fernald (2007) found that Spanish-acquiring preschoolers can use
grammatical gender on determiners (\emph{el} or \emph{la}) to anticipate
the word named in a two-object word recognition task. The information
can also be subcategorical phonetic variation: We found that
English-acquiring toddlers look earlier to a named image when the
coarticulatory formant cues on word \emph{the} predict the noun of the
sentence, compared to tokens with neutral coarticulation (Mahr,
McMillan, Saffran, Ellis Weismer, \& Edwards, 2015).

There is some evidence for lexical competition where children are
sensitive to phonological and semantic similarities among words. Ellis
Weismer, Haebig, Edwards, Saffran, and Venker (2016) showed that
toddlers (14--29 months old) looked less reliably to a named image when
the onscreen competitor was a semantically related word or perceptually
similar image. In Law, Mahr, Schneeberg, and Edwards (2016),
preschoolers (28--60 months old) demonstrated sensitivity to semantic
and phonological competitors in a four-image eyetracking task. Huang and
Snedeker (2011) presented evidence of cascading semantic-phonological
activation in five-year-olds such that for a target word like
\emph{log}, children looked more to an indirect phonological competitor
like \emph{key} (competing through its activation of \emph{lock}) than
they looked to an unrelated image like \emph{carrot}. In contrast to
these studies which all demonstrate interference from similar words,
Mani and Plunkett (2010) demonstrated cross-modal phonological priming
effects in 18-month-olds. In this study, a picture of prime word (e.g.,
cat or teeth) was presented in silence; then two images (e.g., cup and
shoe) were presented, one of which was named (\emph{cup}). Children on
average looked more to the target word (like \emph{cup}) when it was
primed by an image of a phonological neighbor (like \emph{cat}), and the
children performed at chance when the prime was not related to the named
word. Mani, Durrant, and Floccia (2012) found a similar result for
cascading phonological-semantic priming with 24-month-olds: Children
looked more to a target \emph{shoe} compared to a distractor \emph{door}
when primed by an image of \emph{clock}, assumed to activate \emph{sock}
which primed \emph{shoe}.

The above studies involved young children of different ages tested under
different procedures, sometimes in different dialects and languages.
Averaging these results together, so to speak, the studies suggest that
early word recognition demonstrates some hallmarks of adult behavior:
Continuous processing of words, integration of information from
different levels of representation, and the influence of similar,
unspoken words on recognition of a word. Nevertheless, we only have a
fragmented view of how familiar word recognition and lexical competition
develop within children.

One open question is how lexical competition develops within children.
For example, do phonological similar words exert more interference
during word recognition as children grow older? As a guiding hypothesis,
we can think of word learning as a gradual process where familiarity
with a word moves from shallow receptive knowledge to deeper expressive
knowledge. In adult listeners, words compete and inhibit one another, so
that a word is truly ``learned'' and integrated into the lexicon when it
can influence the processing of other words (a line of reasoning
reviewed by Kapnoula, et al., 2015). Increasing sensitivity to similar
sounding words over time would reveal that children improve their
ability to consider multiple candidates in parallel. By studying how
sensitivity to similar-sounding and similar-meaning words develop over
time and within ever-growing vocabularies, this project can reveal how
children come to process words efficiently.

Another avenue for studying word recognition is to examine how listeners
respond to unfamiliar or novel stimuli. A productive line of research
has found that children are sensitive to mispronunciations during word
recognition (e.g., Swingley \& Aslin, 2000, 2002). White and Morgan
(2008) presented toddlers with images of a familiar and novel object,
and children heard a correct production of the familiar object,
mispronunciations of the familiar object of varying severity, or an
unrelated nonword. Toddlers looked less to a familiar word when the
first segment was mispronounced. Moreover, they demonstrated graded
sensitivity such that a 1-feature mispronunciation yielded more looks to
an image than a 2-feature mispronunciation, and a 2-feature
mispronunciation yielded more looks than a 3-feature one. Finally, in
the nonword condition, the children looked more to the novel object than
the familiar one, demonstrating \emph{fast referent selection} as they
associated novel words to novel objects in the moment. A similar pattern
of effects was observed in the mispronunciation study by \citet{MPPaper}
with preschoolers mapping real words to familiar objects, nonwords to
novel objects, and equivocating about mispronunciations of familiar
words.

As with lexical competition, it is unclear how children's responses to
mispronunciations and novel words change over time or how individual
differences among children change over time. For example, do children
become more forgiving of mispronunciations as they mature and learn more
words? We might expect so, as children become more experienced at
listening to noisy, degraded, or misspoken speech.

Another open question involves the development of fast referent
selection. At face value, we might expect a child's ability to associate
new words with unfamiliar objects to be more direct measure of
word-learning capacity than a child's ability to process known words.
Under this assumption, we would expect individual differences in fast
referent selection to be highly correlated with vocabulary growth. But
McMurray, Horst, and Samuelson (2012) propose that the same basic
process is at play in both recognition of familiar words and fast
association of nonwords. In experiments, the observed behaviors are the
same: Children hear a word and direct their attention to an appropriate
referent. This project can tackle these questions by describing how
mispronunciations are processed as children grow older and by examining
whether familiar word recognition and fast referent selection dissociate
and which one is a better predictor of vocabulary growth.

\subsection{Individual Differences in Word
Recognition}\label{individual-differences-in-word-recognition}

We have a rough understanding of the development of word recognition,
and these gaps in knowledge matter because young children differ in
their word recognition abilities. These differences are usually measured
using \emph{accuracy} (a probability of recognizing to a word) or
\emph{efficiency} (a reaction time or some measure of how quickly
accuracy changes over time). These differences are consequential too, as
word recognition differences correlate with other language measures
concurrently, retrospectively, and prospectively.

The best predictor of lexical processing efficiency is concurrent
vocabulary size: Children who know more words look more quickly and
reliably to a named word (e.g., Law \& Edwards, 2015).

This fact deserves a brief reflection: Suppose the information
processing mechanism behind word recognition were just a naïve table
search. Then this finding is somewhat puzzling: Children with larger
lexicons have to find a needle in a larger haystack---yet this apparent
liability is an advantage. That is why the search analogy is naïve. One
explanation follows from the earlier described idea about graded word
learning: Children become better at recognizing words as they learn more
words because they extract regularities and discover similarities among
words and develop more efficient lexical representations---the haystack
develops regularity and becomes easier to search.

Although it is a robust predictor of word recognition, vocabulary size
is nonspecific. For lexical processing dynamics, vocabulary size can be
considered an indicator for the organization and efficiency of a child's
lexicon, but it also correlates with other (meaningful) differences.
Vocabulary is related to differences in speech perception (Cristia,
Seidl, Junge, Soderstrom, \& Hagoort, 2014) and environmental factors
like language input (e.g., Hart \& Risley, 1995; Hoff, 2003). For
instance, measures of speech perception at 6--8 months predict
vocabulary size at 24 months (e.g., Tsao, Liu, \& Kuhl, 2004; Kuhl et
al, 2008), so processing predicts future vocabulary predicts concurrent
processing.

A related complication is the apparent predictive validity of word
recognition measures. Marchman and Fernald (2008) found that vocabulary
size and lexical processing efficiency at age 2 jointly predicted
working memory scores and expressive language scores at age 8. This
result would suggest domain-general processing advantages influence word
learning. Fernald and Marchman (2012) also found that late talkers who
looked more quickly to a named word at 18 months showed larger gains in
vocabulary by 30 months compared to late-talkers who looked more slowly
at 18 months. Weisleder and Fernald (2013) found that lexical processing
and language input at 19 months predict vocabulary size at 25 months and
that lexical processing mediated the effect of language input.

Word recognition efficiency and vocabulary size are interconnected
measures with concurrent and predictive associations. This project can
clarify this relationship by examining the co-development of word
recognition, vocabulary size, and speech perception. In particular, I
will ask how individual differences in word recognition change over time
alongside differences in vocabulary. I can also which features of word
recognition (fast referent selection, lexical competition, etc.) are
most predictive of vocabulary outcomes at age 5. The additional measures
of speech perception can also help clarify the specific effects of
vocabulary size on word recognition.

\subsection{Computational Modeling}\label{computational-modeling}

One way to bolster a theory of word recognition is to implement the
theory in a computational model and simulate human behavior with the
model. If the model can produce responses like those of human listeners,
then the behavioral data support the model and the model's underlying
theory. Models have adjustable parameters, and these generally
correspond to plausible psychological constructs like the inhibition
strength among competing units or a learning rate for modifying
connections. Part of the simulation process therefore is to ask under
what conditions a model simulates human behavior, and then interpret the
simulations in terms of psychological mechanisms.

Here is a hypothetical example of this modeling strategy: Suppose we
want to investigate the finding that a larger vocabulary size predicts
more efficient word recognition. We would ask under which parameters
(i.e., model settings) a model with a large lexicon outperforms one with
a smaller lexicon at word recognition. It might be the case that these
results occur only when parameters are set so that the representations
of speech sounds are comparatively noisier in a model with fewer words.
In this scenario, the models provide a plausible psychological
interpretation for the empirical findings: Children develop better
representations of speech sounds as they learn words, and these better
representations enable more efficient word recognition. Granted, this
example is just a hypothetical case, but it illustrates how I intend to
use computational models as a way to describe word recognition trends
and variation in terms of psychological mechanisms and processing
parameters.

The TRACE model of speech perception and word recognition (McClelland \&
Elman, 1986) is well suited for this kind of simulation work. TRACE can
simulate a dozen or so empirical results from speech perception
literature (Strauss, Harris, \& Magnuson, 2007, Table 1), and it has
been used to simulate word-recognition data from adults (Allopenna,
Magnuson, \& Tanenhaus, 1998), and adults with aphasia (Mirman, Yee,
Blumstein, \& Magnuson, 2011), and toddlers (Mayor \& Plunkett, 2014).

McMurray, Samelson, Lee, and Bruce Tomblin (2010) used TRACE to simulate
word recognition results from adolescents with specific language
impairment (SLI). They mapped certain theories about SLI onto different
model parameters. To test the theory that listeners with SLI have
impairments in acoustic perception, for example, they varied three of
the model's parameters: amount of noise added to the model's mock-speech
input (acoustic resolution), temporal spread of acoustic features in the
input (temporal resolution), and rate of decay in the model's acoustic
feature detectors (perceptual memory). Other theories of SLI (and other
model parameters) provided a closer fit to the observed data than the
perceptual deficit theory. Specifically, lexical decay---``the ability
to maintain words in memory'' (p.~23)---was the most important model
parameter, implying that individual differences in word recognition for
listeners with SLI are rooted in lexical processes, as opposed to
perceptual or phonological ones.

This example shows how simulations with TRACE recast word recognition
performance in terms of psychological processes. For this project, I
will use TRACE to describe how cognitive processes and representations
need to change to simulate the development of word recognition in
preschoolers.

\subsection{Summary}\label{summary-1}

This project studies word recognition in children over three years, so
it will provide the first longitudinal study of word recognition in
preschoolers. Children in this cohort cover a range of vocabulary scores
at Time 1, and this variability allows one to investigate individual
differences in vocabulary and word recognition over time and assess the
predictive value of these measures. Furthermore, this project studies
word recognition in two experimental tasks that can tap into different
aspects of word recognition. Specifically, a four-image experiment with
semantic and phonological foils allows me to study how lexical
competition develops, and a two-image experiment with nonwords and
mispronunciations enables me to study how fast referent selection
develops over time as well. I will use mixed effects modeling to study
not just gross measures of accuracy or interference from distractors,
but the time course and lexical dynamics of word recognition using
growth curve analyses. These empirical models of the longitudinal word
recognition data will be supported by computational models, so that the
developmental changes can be described by plausible psychological
mechanisms of word recognition.

\appendix


\bibliography{assets/book,assets/refs,assets/packages}


\end{document}
